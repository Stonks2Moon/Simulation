% !TEX root =  master.tex
\chapter{Zusammenfassung}

In der vorliegenden Arbeit haben wir uns damit auseinandergesetzt, wie sich Börsenverläufe simulieren lassen können.
Nach der theoretischen Einarbeitung, wurde ein Tool entwickelt, das reale Marktdaten nutzt und damit den Verlauf diverser Wertpapiere simuliert und das Handeln an einer Börse unter Markt ähnlichen Bedingungen ermöglicht. Dabei sei darauf hinzuweisen, dass wir den Anwendungsfall hauptsächlich für Bildungszwecke sehen. Zwar orientieren sich die enthaltenen Szenarien an echten Börsenverläufen, jedoch sind reale Börsenzusammenhänge sehr komplex und die Marktsituationen heute können sich von den von uns gewählten historischen Situationen unterscheiden. Eine in der Simulation gut funktionierende Strategie muss daher nicht zwingend auf einem realen Markt funktionieren. Außerdem distanzieren wir uns von sämtlichen anlageberaterischen Tätigkeiten. Zweck der Anwendung ist ausschließlich das spielerische Erlangen eines besseren Verständnisses über den Aktienmarkt und unterschiedliche Strategien. Dabei ermöglichen wir eine risikofreie Testung von unterschiedlichen Strategien und Ansätzen, da die durch die Simulation durchgeführten Transaktionen keinen Bezug zu realen Währungen und den damit verbundenen finanziellen Verlusten haben. 

\section{Fazit}
Insgesamt sind wir mit der Ausarbeitung und der Arbeit innerhalb unserer Teilgruppe sehr zufrieden. Dennoch hat das Projekt auch bei uns einige Learnings hervorgebracht. Diese liegen im Wesentlichen in der Teamübergreifenden Arbeit und dem Projektgeschehen. 

	\begin{itemize}
		\item Früher POCs und MVPs nutzen: \\
			Dies hätte uns potenziell die Mehraufwände und Einschränkungen durch die Performance und Lastprobleme zwischen der Simulation und der Börse erspart, die sich auch auf die anderen Broker ausgewirkt hat. 
		\item Effektivere Kommunikation in den Abstimmungsmeetings (Agenda, o.\,Ä.)
		\item Klarere Aufgabendefinitionen: \\
			Hätte dazu geführt Aufwände bei der theoretischen Einarbeitung und Implementierungen zu reduzieren.
		\item Früher Feedback vom Kunden: \\
			Auch hier hätte uns schnelleres Feedback unter Umständen Aufwände ersparen können. 
		\item Definierte Ansprechpartner im Fehlerfall
		
	\end{itemize}

In der von uns entwickelten Anwendung sehen wir vor allem den Vorteil, dass diese sehr skalierbar ist. So können einfach weitere Aktien hinzugefügt und simuliert werden. Auch das Ergänzen von weiteren Szenarien ist ohne weitere Implementierungsaufgaben möglich. Außerdem kann unsere Anwendung auch mit höheren Frequenzen, also mehr Trades pro Sekunde, und deutlich größeren Volumina arbeiten.  
