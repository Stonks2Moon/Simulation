% !TEX root =  ../master.tex
\chapter{Konzeption}
\section{Herleitung und Betrachtung verschiedener Szenarien}\label{sec:szenarien}
Damit mit der Simulation der Markt möglich genau widergespiegelt werden kann, steht zunächst die Frage offen,
wie ein typischer Handelstag an der Börse aussieht. Um dies analysieren zu können, wird ein Aktienkurs betrachetet, 
der in den letzten Jahren meistens sehr stabil war und nicht im Rahmen von Marktmanipulation durch Kommunities 
(z.B. GameStop) beeinträchtigt wurde - die SAP Aktie. Nachdem es bei einer Aktie unterschiedliche Szenarien geben, 
wurden die fünf häufigsten Szenarien ausgewählt:
\begin{itemize}
    \item Normaler Handelstags mit einem durchschnittlichen Handelsvolumen
    \item Positive Nachricht an einem Handelstag
    \item Negative Nachricht an einem Handelstag
    \item Niedriges Handelsvomlumen 
    \item Hohes Handelsvolumen
\end{itemize}
Für das erste Szenario wurde der 22.10.2019 ausgewählt. An diesem Tag gab es eine Preisdifferenz von knappen 3\$ 
zwischen Open-Preis und Close-Preis. Außerdem wurden gut 1.000.000 Aktien an der NASDAQ-Börse gehandelt. \\
Am 24.04.2019 wurde die Bilanz des ersten Quatals veröffentlicht. Diese sind deutlich besser ausgefallen als erwartet 
und damit gab es auch eine deutliche Auswirkung auf den Aktienmarkt. Die Aktie hatte einen Unterschied von 8\$ zwischen 
Open und Close und ein Handelsvolumen von knapp 4.700.000 gehandelten Aktien. \\
Das dritte Szeanrio ist der erste Handelstag nach den Veröffentlichung der Quatalszahlen (26.10.2020). Aus diesen war 
herauszulesen, dass es einen deutlichen Zurückgang des Gewinnes gab und danach ist der Aktienkurs von 149\$ (Freitag Close) 
auf 115\$ (Montag Close) gefallen. Zusätzlich gab es ein sehr hohes Handelsvolumen von 11.000.000 Aktien. \\
Als Beispieltag für ein gering gehandeltes Volumen wurde Weihnachten 2019 genommen. An diesem Tag wurden nur 117.000 Aktien gehandelt 
und es gab eine Preisdifferenz von Open und Close von 1\$. \\
In den letzten Jahren war der Handelstag mit dem größten Handelsvolumen der 26.10.2020. Um für zwei Szenarien nicht den gleichen 
Tag zu nehmen wurde der 28.10.2020 genommen, an dem das Handelsvolumen noch bei 5.500.000 Aktien wegen den Auswirkungen der 
Quatalszahlen war. \\
Die Daten für alle Szenarien sind von der NASDAQ-Börse aus New York und wurden auf Minutenbasis gespeichert. 
\section{Umwandlung von Szenarien in Orders}
Nachdem die Daten der verschiedenen Szenarien in JSON-Format vorliegen müssen diese zunächst noch angepasst werden. Die 
Szenarien sollen später für verschiedene Aktien verwendet werden. Um dies zu ermöglichen werden die Aktienkurse nicht in 
absoluten Zahlen gespeichert, sondern es wird die Änderungsrate des Aktienkurses im Vergleich zum Aktienkurs eine Minute davor 
ausgerechnet und abgespeichert. Dadurch ist es möglich das Szenario mit beliebigen Aktien durchzuspielen. \\
Im nächsten Schritt müssen anhand der Daten Orders erstellt werden, um die Aktienkurs zu simulieren. Dafür muss es jeweils eine Limit-Buy-Order 
und eine Limit-Sell-Order zum gleichen Preis geben, damit die Orders matchen und es einen neuen Aktienpreis gibt. Der Preis für die 
Limit-Orders wird aus dem aktuellen Aktienpreis und der Änderungsrate berechnet. Alleine mit diesem Vorgang würde es schon reichen 
den Aktienkurs eines Szenarien durchzuspielen, denn es würden sich immer direkt die beiden Orders matchen und der neue Aktienpreis 
wäre gesetzt. Nachdem aber durch die Simulation der gesamte Markt widergespiegelt werden soll, müssen mehr Orders gestellt werden, damit 
andere Markteilnehmer auch eine Chance haben eine Aktie zu kaufen. Zusätzlich darf es nicht passieren, dass andere Marktteilnehmer 
den Aktienkurs mit \enquote{komischen} Orders beeinflussen. Damit diese Probleme gelöst werden, müssen immer eine gewisse Anzahl an Orders 
im Orderbuch stehen. Diese Orders sollten immer relativ nahe am aktuellen Kurs sein, damit andere Marktteilnehmer zu normalen Preise die 
Aktie kaufen können und Limit-Orders mit \enquote{komischen} Preisen nicht zum matchen kommen. \\
Für die Befüllung des Orderbuches werden fünf Buy- und fünf Sell-Limit-Orders jeweils ein paar Prozent über und unter dem aktuellen 
Preis gesetzt. Dabei wird eine Aufteilung von 80 zu 20 Prozent verwendet, also 80 Prozent des Ordersvolumens einer Minute werden in jeweils 
in die beiden matchenden Orders aufgeteilt und die restlichen 20 Prozent werden für Orders zum Befüllen des Orderbuches benutzt.
\section{Benachrichtigungen über Marktereignisse}
Nachdem bei einer Simulation nicht nur die Daten simuliert werden sollen, sondern ein normaler Handelstag, wäre es sinnvoll auch Nachrichten 
zu dem Unternehmen der Aktie zu veröffentlichen. Dabei können automatisiert Nachrichten über Telegram versendet werden, wenn der Kurs beispielsweise 
an einer bestimmten Stelle angekommen ist. Für unsere Szenarien wäre zum Beispiel Nachrichten wie: \enquote{Unternehmen XY veröffentlicht Quatalszahlen. 
Gewinneinbruch von 20\%!} sinnvoll zu veröffentlichen. Diese Nachrichten werden dann durch einen Nachrichtenbot versendet und anschließend sollte zu 
sehen sein, dass der Aktienkurs nach unten geht. Diese Möglichkeit hat den Vorteil für die Marktteilnehmner, dass sie reagieren können und nicht 
rein spekulativ handeln müssen.
\section{Berücksichtigung des Handelns anderer Marktteilnehmern}
TODO: eventuell weglassen, wenn Geschäftskunden keine Auswirkung haben.