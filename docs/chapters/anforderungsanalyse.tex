% !TEX root =  ../master.tex
\chapter{Anforderungsanalyse}

In diesem Kapitel werden zuerst die funktionalen und nicht-funktionalen Anforderungen der Simulation erläutert. Anschließend werden diese Anforderungen übersichtlich zusammengefasst. Im Anschluss werden kurz die Anforderungen aus unserer Sicht zu den anderen Teilprojekten erläutert.
\section{Funktionale Anforderungen}
	\begin{itemize}
		\item Anmeldung: \\
			Damit das Marktgeschehen nicht durch jede beliebige Person beeinflusst werden kann, soll die Anwendung durch ein Login geschützt werden. So können nur dedizierte Personen Szenarien auswählen und wichtige Marktfaktoren beeinflussen. Denkbar wäre hierdurch auch das Verteilen von Berechtigungen.
			Beispielsweise wäre es möglich, lesende Operationen wie das gehandelte Volumen für alle einsehbar zu machen, während schreibende Operationen, die das Marktgeschehen beeinflussen könnten, nur ausgewählte Personen tätigen können.
		
		\item Auswahl vordefinierter Szenarien: \\
			Es soll eine Liste an vordefinierten Szenarien geben. Diese Szenarien können bestimmte Ereignisse sein, wie der Rücktritt einzelner Vorstandsmitglieder, auf die der Markt in einem vorher definierten Rahmen reagiert. Diese Reaktion soll automatisch mit dem Beginn des nächsten Szenarios durch die Simulation auf den Markt übertragen werden.
			
		\item Übersicht über die gerade gehandelten Trades: \\
			Es sollte eine grafische Übersicht geben, die einen Überblick über das durch die Simulation gehandelten Volumina und Preise verschafft.
			So soll ein besserer Einblick über die aktuelle Simulationsaktivität gegeben werden.
			
		\item Kommunikation von Events: \\
			Bestimmte Ereignisse beeinflussen das Marktgeschehen. Damit Privatkunden und Unternehmen, die an der Börse handeln, entsprechend reagieren können, müssen sie über solche Ereignisse aktiv informiert werden. Ob und wie sie auf diese Nachrichten reagieren bleibt ihnen selbst überlassen.
			
		\item Reaktion auf Marktgeschehen durch Privat- und Unternehmensbroker: \\
			Um die Simulation möglichst realistisch zu gestalten, muss auch das Verhalten und die Trades der anderen Marktteilnehmer berücksichtigt werden. Wichtig ist hier jedoch, dass nicht eine einzelne Transaktion eines Privatkunden schon die Marktpreise zum Schwanken bringt.
			Stattdessen sollen Transaktionen erst eine kritische Masse erreichen müssen, um wirklich auf dem Markt ins Gewicht zu fallen. 
			
		\item Anpassung der Dauer eines Szenarien: \\
			Um das schnelle Testen von Anlagestrategien zu ermöglichen, soll die Länge der Szenarien skalierbar sein.
	\end{itemize}
	
\section{Nicht-Funktionale Anforderungen}
		\begin{itemize}
		\item Geschwindigkeit: \\
			Für Nutzer ist die Geschwindigkeit besonders wichtig. Lang ladende Anwendungen sind nervig in der Benutzung und demotivieren die Nutzung der Anwendung. Über lange Ladezeiten beschweren sich in einer Umfrage mehr als 70\%, wobei 25\% aller Nutzer bereits bei 4 Sekunden die Seite verlassen. Aus diesem Grund sollte die Ladezeit gering gehalten werden.
			
		\item Skalierbarkeit: \\
			Um das bereits angesprochene hohe Ordervolumen zu realisieren, muss die Anwendung gut skalieren können.
			Bei zunehmender Zahl der Transaktionen oder Wertpapieren darf die Reaktion und die Performance der Anwendung nicht beeinträchtigt werden. 
	\end{itemize}

\section{Zusammenfassung}
	\begin{table}[ht!]
		\centering
		\begin{tabularx}{.8\textwidth}{l|X}
			Nr.     & Beschreibung                              \\\hline
			F1      & Am System soll eine Anmeldung möglich sein                  \\
			F2		& Es soll eine Szenarienauswahl geben \\
			F3      & Nutzer sollen eine Übersicht über die aktuelle Simulationsaktivität bekommen  \\
			F4      & Events, die das Marktgeschehen beeinflussen sollen aktiv kommuniziert werden  \\
			F5      & Es soll auf das Marktgeschehen durch andere Privat- und Unternehmensbroker reagiert werden  \\
			F6      & Die Zeiteinheiten und somit die Dauer eines Szenarien sollte angepasst werden können  \\\hline
			NF1     & Anwendung lädt schnell                    \\
			NF2     & Die Anwendung ist skalierbar                \\
		\end{tabularx}
		\caption{Zusammengefasste Anforderungen}
		\label{tab:anforderungen}
	\end{table}



\section{Anforderungen an andere Teams}\label{sec:otherTeams}
	Im Falle der Simulation bestehen keine Abhängigkeiten zu anderen Brokern. Bei den Privatkunden und Unternehmenskunden handelt es sich auch um Broker.
	Diese kaufen und verkaufen an der Börse Wertpapiere. Genauso wie bei der Simulation handelt es sich damit um Broker bzw. Clients der Börse, die sich lediglich in ihrer Zielgruppe und dem Zweck unterscheiden. Somit bleibt als einzige Schnittelle die Börse. Dabei wurden folgende Anforderungen aufgestellt:
	
	\textbf{GET-Requests}
		\begin{itemize}
			\item aktueller Preis eines Wertpapiers
			\item im Handel befindliche Wertpapiere
		\end{itemize}
	\textbf{POST-Requests}
		\begin{itemize}
			\item Order einstellen
		\end{itemize}
	\textbf{Aktive Benachrichtigungen}
		\begin{itemize}
			\item Buchung der Order
			\item Handel unterbrochen
		\end{itemize}
