% !TEX root =  ../master.tex
\chapter{Theoretische Herleitungen}
\section{Betrachtung von Marktfaktoren}

Der erste verfolgte Ansatz, um einen realistischen Aktienmarkt zu simulieren, stellte das Fama-French-Dreifaktorenmodell dar. Bei diesem Modell ist die individuelle Wertpapierrendite, sowohl von individuellen Marktfaktoren, als auch von Marktfaktoren, die auf den gesamten betrachteten Markt wirken, erklärt. \\
Dieses Modell hat sich im Rahmen der Simulation als ungeeignet erwiesen, jedoch sind anhand der verwendeten Marktfaktoren einige Marktentwicklungen zu erklären, die entsprechend die Wertentwicklung eines Wertpapiers beeinflussen.\\
Darauffolgend ist die Simulation mithilfe von praxisnahen Szenarien durchgeführt worden, welche im folgenden Kapitel näher dargestellt sind. An dieser Stelle sei darauf hingewiesen, dass bei der Erklärung der unterschiedlichen Szenarien teilweise auf das Fama-French-Dreifaktorenmodells referenziert und die Szenarien damit begründet werden.

Das Fama-French-Dreifaktorenmodell erklärt die Aktienrendite ($r_i$) anhand von den Aktien-individuellen Marktfaktoren:
\begin{itemize}
	\item Aktienmarktrisikoprämie : ($R_M$)
	\item Größe des Unternehmens : (${SMB}$)
	\item Buchwert-Kurs-Verhältnis : ($HML$)
	\item Risikofreier Zinssatz : ($r_F$)
\end{itemize}

Bestimmung der potentiellen Aktienrendite erfolgt dem (vereinfachten) Modell nach mit folgender Formel:

$$ r_i = r_F + R_M + \beta_{iSMB} \cdot SMB + \beta_{iHML} \cdot HML$$

($r_F$), ($R_M$), ($\beta_{iSMB}$), ($\beta_{iHML}$) sind Faktoren, die auf den gesamten Markt wirken. Je nach Marktentwicklung und -situation können die $\beta$-Werte positiv oder negativ sein.

Der risikofreie Zinssatz ($r_F$) ist in Europa zu Zeiten der Nullzinspolitik der EZB faktisch zu vernachlässigen.

Die Aktienmarktrisikoprämie ($R_M$) ist abhängig von den vorherrschenden Risiken, die auf den Markt wirken. 

Ein positives ($\beta_{iSMB}$) wirkt sich bei einer kleinen Marktkapitalisierung positiv auf den Aktienkurs aus, eine große Marktkapitalisierung entsprechend negativ. Analog ist dieser Effekt bei einem negativen ($\beta_{iSMB}$) gegensätzlich.

Ein positives ($\beta_{iHML}$) wirkt sich bei einem steigenden Kurs-Buchwert-Verhältnis positiv aus, bei einem sinkenden entsprechend negativ. Analog ist dieser Effekt bei einem negativen ($\beta_{iHML}$) gegensätzlich.

Hierbei ist hervorzuheben, dass sich die Aktienrendite über beide Faktoren ($SMB$ : Marktkapitalisierung = Aktienkurs $\cdot$ Umlaufende Aktien) und \\
($HML$ : Kurs-Buchwert-Verhältnis = $\frac{\text{Kurswert der Aktie}}{\text{Buchwert der Aktie}} $) auf den Aktienkurs bezieht. Daher könnte hierbei für jeden individuellen Aktienkurs eine potentielle Aktienrendite errechnet werden, worauf Kauf- oder Verkaufsentscheidungen gegründet sein können. Jedoch können diese Entscheidungen nicht konsistent die Marktentwicklung simulieren, da sich der Markt in vielen Aspekten (besonders in außergewöhnlichen Situationen, z.B. einer weltweiten Pandemie) unberechenbar verhält und eine umfassendere Simulation nicht im Rahmen dieses Projekts umsetzbar gewesen wäre.\\
Aus diesen Erkenntnissen ist die Entscheidung gefallen, vergangene Kurstage eines DAX-Unternehmens zu betrachten, welche exemplarisch spezifische Situationen darstellen sollen.

\section{Unterschiedliche Szenarien}

Um eine möglichst realistische Simulation des Wertverlaufs einer Aktie darzustellen wurden fünf verschiedene Markttage der SAP-Aktie aus den letzten zwei Jahren betrachtet. Die Entwicklung an diesen Tagen wurde als Vorlage für die Simulation genommen.

Folgend sind die fünf ausgewählten Markttage aufgeführt, wobei jeweils die Begründung für den explizit gewählten Tag gegeben ist.
\begin{enumerate}
	\item Rapider Kursfall nach Firmenübernahme : 26.10.2020
	
		Dieses Szenario ist einem Kurstag entnommen, bei welchem der Markt auf die große Übernahme reagierte. Hierbei handelte es sich um einen Montag, nachdem die Information der Übernahme freitags zuvor nach Handelsschluss öffentlich gemacht worden war. Bei einer solchen großen Übernahme wird der Buchwert einer Aktie negativ beeinflusst, was den nachhaltigen Einbruch des Aktienkurses erklären könnte. Dies ist ebenfalls nach dem Fama-French-Dreifaktorenmodell plausibel, wenn das ($\beta_{iHML}$) einen negativen Wert annimmt (und die Aktienrendite konstant bleibt).
	
	\item Geringes gehandeltes Volumen : 01.10.2020
	
		Innerhalb dieses Markttages wurden besonders wenig Aktienanteile gehandelt, wobei auch der Aktienkurs relativ konstant verlief. In diesem Szenario der Simulation hätten einzelne Akteure den Preis der Aktien massiv beeinflussen können.  
		
	\item Hohes gehandeltes Volumen : 19.06.2020
	
		Dieser betrachtete Markttag stellt das Gegenteil zu dem vorherigen Markttag dar. Viele Käufer und Verkäufer handelten, weswegen einzelne Akteure keinen wesentlichen Einfluss auf den Kurswert ausüben konnten.
	
	\item Positive Nachricht : 24.04.2019
	
		Innerhalb dieses Markttages sind besser ausgefallene Gewinne aus dem vorherigen Quartal veröffentlicht worden, was zu einem rasanten Anstieg des Kurswerts an diesem Tag führte. Damit verringerte sich das Kurs-Buchwert-Verhältnis, wobei - gleichbedeutend mit einem negativen ($\beta_{iHML}$) - der Aktienkurs (bei gleicher Aktienrendite) stieg.
	
	\item Normaler Handelstag mit durchschnittlichem Volumen : 22.10.2019
	
		Dieser Tag wurde als letztes Szenario betrachtet, um einen möglichst durchschnittlich verlaufenden Markttag darstellen zu können.
	
\end{enumerate}

